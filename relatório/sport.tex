%%%%%%%%%%%%%%%%%%%%%%% file typeinst.tex %%%%%%%%%%%%%%%%%%%%%%%%%
%
% This is the LaTeX source for the instructions to authors using
% the LaTeX document class 'llncs.cls' for contributions to
% the Lecture Notes in Computer Sciences series.
% http://www.springer.com/lncs       Springer Heidelberg 2006/05/04
%
% It may be used as a template for your own input - copy it
% to a new file with a new name and use it as the basis
% for your article.
%
% NB: the document class 'llncs' has its own and detailed documentation, see
% ftp://ftp.springer.de/data/pubftp/pub/tex/latex/llncs/latex2e/llncsdoc.pdf
%
%%%%%%%%%%%%%%%%%%%%%%%%%%%%%%%%%%%%%%%%%%%%%%%%%%%%%%%%%%%%%%%%%%%

\documentclass[runningheads,a4paper]{llncs}

\usepackage[portuguese]{babel}
\usepackage[utf8]{inputenc}
\usepackage{amssymb}
\setcounter{tocdepth}{3}
\usepackage{graphicx}
\usepackage{verbatim}

\usepackage{url}
\newcommand{\keywords}[1]{\par\addvspace\baselineskip
\noindent\keywordname\enspace\ignorespaces#1}

\begin{document}

\mainmatter

\title{Programação em Lógica\\Trabalho prático 2\\Calendarização de Competição Desportiva \linebreak}

\titlerunning{Calendarização de Competição Desportiva}

\author{\textbf{Turma 3MIEIC02 - Grupo 13:}\\ Hugo Ari Rodrigues Drumond - 201102900 \\ João Alexandre Gonçalinho Loureiro - 200806067 \linebreak}

\authorrunning{Calendarização de Competição Desportiva}

\institute{Faculdade de Engenharia da Universidade do Porto \\ Rua Roberto Frias, 4200-465 Porto, Portugal}

\toctitle{Trabalho prático para Programação em Lógica}
\tocauthor{Authors' Instructions}
\maketitle

\begin{abstract}
The abstract should summarize the contents of the paper and should
contain at least 70 and at most 150 words. It should be written using the
\emph{abstract} environment.
\keywords{We would like to encourage you to list your keywords within
the abstract section}
\end{abstract}
\newpage

\section{Introdução}
%Descrição dos objetivos e motivação do trabalho, descrição do
%problema, (idealmente, referência a outros trabalhos na mesma área + descrição muito
%sucinta da aproximação utilizada, salientado as principais diferenças em relação aos outros
%trabalhos), e estrutura do resto do artigo.
da

\section{Descrição do problema}
%Descrever sucintamente o problema de
%otimização ou decisão em análise.
da

\section{Ficheiros de Dados}
%Descrever os problemas a resolver e o conteúdo dos
%respetivos ficheiros de dados. Devem ser construídos pelo menos três problemas distintos
%em ficheiros de texto separados. (Idealmente: utilização de datasets usados noutros
%trabalhos de outros autores).
da

\section{Variáveis de Decisão}
%Descrever as variáveis de decisão e os seus domínios.
da

\section{Restrições}
%Descrever as restrições rígidas e flexíveis do problema e a sua
%implementação utilizando o SICStus Prolog.
da

\section{Função de Avaliação}
%Descrever, quando for o caso, a forma de
%avaliar a solução obtida e a sua implementação utilizando o SICStus Prolog.
da

\section{Estratégia de Pesquisa}
%Descrever a estratégia de etiquetagem
%(“labeling”) implementada, nomeadamente no que diz respeito à ordenação de variáveis e
%valores. Deve também ser descrita a forma de implementação desta estratégia utilizando a
%linguagem.
da

\section{Visualização da Solução}
%Explicar os predicados solução em modo de texto (obrigatório) ou modo gráfico (opcional).
da

\section{Resultados}
%Demonstrar exemplos de aplicação em casos práticos com
%diferentes complexidades e analisar os resultados obtidos. Devem ser utilizadas formas
%convenientes para apresentação dos resultados (tabelas e/ou gráficos). (Idealmente:
%comparação de resultados com outros autores/trabalhos.)
da

\section{Conclusões e Perspectivas de Desenvolvimento}
%Que conclusões retiram deste projeto? O que mostram os resultados obtidos? Quais as vantagens
%e limitações da solução proposta? (Idealmente: conclusão sobre a comparação com os
%resultados obtidos por outros autores/trabalhos.) Como poderiam melhorar o trabalho
%desenvolvido?
da

\end{document}
