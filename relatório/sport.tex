%%%%%%%%%%%%%%%%%%%%%%% file typeinst.tex %%%%%%%%%%%%%%%%%%%%%%%%%
%
% This is the LaTeX source for the instructions to authors using
% the LaTeX document class 'llncs.cls' for contributions to
% the Lecture Notes in Computer Sciences series.
% http://www.springer.com/lncs       Springer Heidelberg 2006/05/04
%
% It may be used as a template for your own input - copy it
% to a new file with a new name and use it as the basis
% for your article.
%
% NB: the document class 'llncs' has its own and detailed documentation, see
% ftp://ftp.springer.de/data/pubftp/pub/tex/latex/llncs/latex2e/llncsdoc.pdf
%
%%%%%%%%%%%%%%%%%%%%%%%%%%%%%%%%%%%%%%%%%%%%%%%%%%%%%%%%%%%%%%%%%%%

\documentclass[runningheads,a4paper]{llncs}

\usepackage[portuguese]{babel}
\usepackage[utf8]{inputenc}
\usepackage{amssymb}
\setcounter{tocdepth}{3}
\usepackage{graphicx}
\usepackage{verbatim}

\usepackage{url}
\newcommand{\keywords}[1]{\par\addvspace\baselineskip
\noindent\keywordname\enspace\ignorespaces#1}

\begin{document}

\mainmatter

\title{Programação em Lógica\\Trabalho prático 2\\Calendarização de Competição Desportiva \linebreak}

\titlerunning{Calendarização de Competição Desportiva}

\author{\textbf{Turma 3MIEIC02 - Grupo 13:}\\ Hugo Ari Rodrigues Drumond - 201102900 \\ João Alexandre Gonçalinho Loureiro - 200806067 \linebreak}

\authorrunning{Calendarização de Competição Desportiva}

\institute{Faculdade de Engenharia da Universidade do Porto \\ Rua Roberto Frias, 4200-465 Porto, Portugal}

\toctitle{Trabalho prático para Programação em Lógica}
\tocauthor{Authors' Instructions}
\maketitle

\begin{abstract}
The abstract should summarize the contents of the paper and should
contain at least 70 and at most 150 words. It should be written using the
\emph{abstract} environment.
\keywords{We would like to encourage you to list your keywords within
the abstract section}
\end{abstract}
\newpage

\section{Introdução}

%Descrição dos objetivos e motivação do trabalho, descrição do
%problema, (idealmente, referência a outros trabalhos na mesma área + descrição muito
%sucinta da aproximação utilizada, salientado as principais diferenças em relação aos outros
%trabalhos), e estrutura do resto do artigo.
O trabalho tem como objetivo, usar os conceitos de restrições da Programação em Lógica, para fazer a Calendarização de uma Competição Desportiva. A abordagem PLR é muito mais eficiente porque segue o princípio de restringir e só depois gerar ao invés da abordagem "padrão" do prolog, gerar e testar. A forma de atacar o problema seguindo este novo princípio, é feita da seguinte maneira: 1º define-se um domínio para as variáveis do problema; 2º aplica-se restrições a essas variáveis que irão diminuir o domínio do problema, feito através de consistência dos arcos; 3º Pesquisa da solução de diferentes maneiras, por exemplo: instanciar o valor mais a esquerda e ir testando as restrições e selecionando novas instanciações e testando novamente,etc. A programação em lógica com restrições é do nosso agrado pois permite resolver problemas complexos de forma eficiente através de conceitos matemáticos: domínios, expressões sobre as variáveis restringem o domínio e procura da solução dado o domínio reduzido. Não nos foi possível dialogar com os outros grupos visto que tivemos de mudar de trabalho à última da hora. Isto deveu-se a algumas dúvidas de interpretação do enunciado de Gestão de Servidores Cloud, que nos impossibilitaram de progredir no trabalho. Mandámos dois emails ao professor Tiago Pinto Fernandes  só que no primeiro(12/05) não ficámos esclarecidos e não recebemos resposta ao segundo(13/05).

\section{Descrição do problema}
%Descrever sucintamente o problema de
%otimização ou decisão em análise.
O Calendarização tem as seguintes características:
\begin{description}
\item[Tipo de competição:] Round Robin.
\item[Jornadas:] Deve-se evitar que uma equipa jogue em casa mais que uma vez numa dada jornada. Cada equipa deve alternar ao máximo os jogos em casa e fora, se joguei nesta jornada em casa na próxima devo jogar fora.
\item[Voltas:] Numa competição com duas voltas, o calendário da 2ª volta é idêntico ao da 1ª, mantendo-se a ordem dos jogos e alternando a situação casa/fora. Cada equipa joga contra cada uma das restantes equipas uma vez em cada uma das voltas da competição.
\item[(Opcional) Distâncias:] Se a competição for realizada em territórios vastos o critério distância deve prevalecer sobre o objetivo alternância casa/fora. Tal pode ser feito se minimizarmos as deslocações de uma equipa, juntando os jogos a realizar fora num dado local em jornadas consecutivas.
\item[Balanceamento:] não pode haver mais do que X clubes grandes a jogar em casa numa dada jornada.
\end{description}
Iremos ter de transformar estas condições para restrições de modo a construir um calendário que as respeite.

\section{Ficheiros de Dados}
%Descrever os problemas a resolver e o conteúdo dos
%respetivos ficheiros de dados. Devem ser construídos pelo menos três problemas distintos
%em ficheiros de texto separados. (Idealmente: utilização de datasets usados noutros
%trabalhos de outros autores).
Foram criados três ficheiros com os seguintes campos: numeroDeEquipas, numeroDeVoltas, escolha, equipa1, equipa2, jornadamin, jornadamax. Escolha pode ser \textit{1} ou \textit{0}. Não nos foi possível partilhar ficheiros com os outros grupos, devido ao problema que referimos na introdução.

\section{Variáveis de Decisão}
%Descrever as variáveis de decisão e os seus domínios.
da

\section{Restrições}
%Descrever as restrições rígidas e flexíveis do problema e a sua
%implementação utilizando o SICStus Prolog.
A única restrição flexível do nosso problema é a das jornadas mínima e máxima para a ocorrência de um jogo entre determinadas equipas, o valor dessas equipas deve estar entre 1 e o numero de equipas. Caso a escolha seja \textit{0} então esta restrição não é imposta no nosso problema. No nosso problema temos diversas restrições fixas: restrições para os jogos das equipas, restrições para as jornadas, e para cada jogo. ConstraintForTeam, ConstraintForRound e ConstraintForGame. Foram criados vários predicados para ir buscar os valores que nos interessavam de Vars para depois se aplicarem as restrições necessárias. Por exemplo: temos um predicado chamado getTeamList, cujo objetivo é ir buscar a sublista das equipas que jogam contra uma dada equipa; getRoundList, vai buscar a lista das equipas que jogam na jornada n. Após irmos buscar essas listas aplicamos certas restrições.

%\section{Função de Avaliação}
%Descrever, quando for o caso, a forma de
%avaliar a solução obtida e a sua implementação utilizando o SICStus Prolog.
%da

\section{Estratégia de Pesquisa}
%Descrever a estratégia de etiquetagem
%(“labeling”) implementada, nomeadamente no que diz respeito à ordenação de variáveis e
%valores. Deve também ser descrita a forma de implementação desta estratégia utilizando a
%linguagem.
A técnica de labeling utilizada é a padrão. Instanciação lado esquerdo, teste de restrições, instanciação, teste, etc. Até chegar a uma solução. É passado para o labeling uma lista chamada Vars. Nesta lista encontra-se os jogos de cada equipa em cada volta. NumeroEquipas*Nrondas*Voltas.

\section{Visualização da Solução}
%Explicar os predicados solução em modo de texto (obrigatório) ou modo gráfico (opcional).
Cada solução possível é imprimida para o ecrã, após o labeling. Isto é feito através da regra, print(Vars,NTeams,NRounds,1).

\section{Resultados}
%Demonstrar exemplos de aplicação em casos práticos com
%diferentes complexidades e analisar os resultados obtidos. Devem ser utilizadas formas
%convenientes para apresentação dos resultados (tabelas e/ou gráficos). (Idealmente:
%comparação de resultados com outros autores/trabalhos.)
da

\section{Conclusões e Perspectivas de Desenvolvimento}
%Que conclusões retiram deste projeto? O que mostram os resultados obtidos? Quais as vantagens
%e limitações da solução proposta? (Idealmente: conclusão sobre a comparação com os
%resultados obtidos por outros autores/trabalhos.) Como poderiam melhorar o trabalho
%desenvolvido?
da

\end{document}
